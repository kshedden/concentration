In this chapter we explore nonparametric quantile regression as a tool
for understanding conditional relationships between the full
distribution of an outcome variable $y\in {\cal R}$ and one or more
explanatory variables $x \in {\cal R}^d$.

The work presented in this chapter is motivated by questions arising
in human biology about the relationship between anthropometry (body
size) and blood pressure.  We provide more background in ###, but
briefly, while it is well-known that individuals who are taller and/or
have greater body mass tend to have greater current blood pressure,
these relationships are modified in complex ways by other factors,
including by sex and age.  Further, when considering longitudinal
trajectories of anthropometry variables as predictors of subsequent
blood pressure (e.g.\ looking at anthropometry over the developmental
span from birth to adulthood), there may be complex
inter-relationships among anthropometry variables taken at different
ages, and among different types of anthropometry variables (e.g.\
height and adiposity).  Finally, factors associated with either high
or low blood pressure may operate differently than factors associated
with median or mean blood pressure.  This motivates our focus on
conditional quantiles in this and the subsequent chapter of the
thesis.

The quantile nearest neighbor (QNN) algorithm presented in [ref] is a
local nonparametric estimation procedure for estimating conditional
quantiles $Q_y(p; x)$ of a quantitative variable $y$ given one or more
explanatory variables $x$.  For a probability point $p \in (0, 1)$,
the QNN procedure allocates a parameter $\theta_i$ to every observed
value $y_i$, and uses the check function $\rho_p(y) = |p - {\cal
I}_{y<0}|$ to measure the fit between the $y_i$ and the $\theta_i$.  A
form of neighborhood regularization based on a graph $G = G(X)$ is
used to control bias and variance.  If two observations indexed $i$
and $i^\prime$ are neighbors in $G$, then a term $\lambda |\theta_i
- \theta_{i^\prime}|$ is included as an additive penalty with a
regularization parameter $\lambda > 0$.  The construction of $G$ is
based on nearest neighbors determined through the covariates in $X$.

QNN seems to have been motivated with prediction as a main goal,
whereas our goal here is to provide interpretable insight into data
and the population that they represent.  Doing so is challenging for
at least three reasons.  First, as considered in more detail below,
the QNN procedure, like other local estimation procedures, exhibits a
bias/variance tradeoff.  In general, estimates of conditional
quantiles based on QNN will exhibit high variance that is related more
to the neighborhood size than to the overall dataset size.  Also, the
variance of these estimates tends to be greater than their squared
bias.  In other words, the output of QNN is ``noisy'', especially for
extreme probability points $p$.  Second, the QNN algorithm provides a
voluminous output consisting of an estimated conditional quantile
$\theta_i$ for each observation $y_i$, which are somewhat difficult to
directly interpret in aggregate.  Third, like other local regression
procedures, QNN suffers from the ``curse of dimensionality'', limiting
its use with multiple covariates.  For these three reasons, in this
chapter our main goal is to develop four ways to post-process the
results of QNN to provide insight about a population of scientific
interest.  Throughout the discussion we use for illustration two
datasets relating to the relationship between human anthropometry and
blood pressure.

\subsection{QNN bias and variance}

Like other local regression procedures such as the more familiar local
linear regression, QNN trades off bias and variance based on the
neighborhood size.  However unlike local regression procedures based
on least squares, reducing the neighborhood size or allowing the
regularization parameter $\lambda$ to shrink to zero does not produce
unbiased estimates of the conditional quantiles in QNN.  In fact, as
$\lambda$ approaches zero, the estimated $\theta_i$ approach $y_i$,
since this minimizes the check function for any $p$.  Each $y_i$ can
be interpreted as a median-unbiased estimate of its own true median,
but if $p\ne 1/2$, setting $\theta_i = y_i$ leads to biased quantile
estimates.  Therefore, the bias/variance tradeoff in QNN is more
complicated than in local least squares regression.  Following the
guidance in [ref] we set $\lambda=0.1$ and build $G$ using
neighborhoods of size $5$.  In this section, we present simulation
studies that further explore the bias and variance properties of the
QNN algorithm.

[...]

\subsection{Interpreting QNN estimates using dimension reduction regression}

After choosing a set of probability points, say $p=[0.1, 0.2, \ldots,
0.9]$, we can use the QNN approach to estimate each conditional
quantile in $p$ for all $n$ observations in our sample of data.  This
yields a $n\times q$ matrix $Q$ (here $q=9$) containing the estimates
$\hat{Q}_p(y_i; x_i)$.  We note that $Q$ is a function of both the
response variable $Y\in {\cal R}^n$ and the matrix $X\in {\cal
R}^{n\times d}$ of explanatory variables.  A natural goal is to assess
how the quantiles vary with $X$.  Since the estimated quantiles in $Q$
were explicitly constructed using $X$, there is an inbuilt
relationship between $Q$ and $X$.  The question of interest is what
form of relationship exists in the population -- that is, were we to
have the true quantiles $Q_0$, what would be the relationship between
$X$ and $Q_0$?

Since each row of $Q$ and of $X$ corresponds to an observation, and
both $Q$ and $X$ have multiple columns, it is natural to consider
multivariate regression.  In this section, we consider the classical
approach of canonical correlation analysis (CCA), and then we consider
a more modern approach that may better capture non-linear
relationships.

CCA seeks to find vectors $\beta_1, \ldots, \beta_r$ and
$\eta_1, \ldots, \eta_r$ such that (i) the vectors $X\beta_j$ are
pairwise uncorrelated, (ii) the vectors $Q\eta_j$ are pairwise
uncorrelated, and (iii) subject to (i)-(ii), the correlation between
each $X\beta_j$ and $Q\eta_j$ is maximized.  The solution to this
constrained optimization problem is efficiently obtained using the
singular value decomposition (SVD).

Since CCA is defined in terms of correlation coefficients which are
scale-invariant, the magnitude of the vectors $\beta_j$ and $\eta_j$
are not well-defined, and by convention are scaled so that
$\|\beta_j\| = 1$ and $\|\eta_j\| = 1$ for all $j$.  The variance for
a CCA-optimal variate, say ${\rm var}(Q\eta_j$), may be very small
compared to ${\rm var}(Q\tilde{\eta})$ for some other unit vector
$\tilde{\eta}$.  That is, CCA may find relationships for which the
correlations between the $X$ and $Y$ variates are strong, but where
either variate is a very small part of the overall variation (for $X$
or $Y$ respectively).  To address this, we can pre-process the data
with Principal Components Analysis (PCA).  In our setting this concern
mainly arises with $Q$, so focusing on that side of the correlation,
we first project $\tilde{Q} \equiv QP$ onto a given number $m$ of
principal components, then conduct CCA using $X$ and $\tilde{Q}$
instead of using $X$ and $Q$.  If $\tilde{\eta}_k$ is a CCA
coefficient vector for $\tilde{Q}$ in this analysis, then
$\eta_k \equiv P\eta_k$ represents the same coefficient vector
expressed with respect to the original $Q$, for interpretability.

The loading vectors $\eta_k$ for $Q$ correspond to factors in the
space of quantile functions that are predictable from $X$.  To further
aid interpretation, we note that if $\eta_k \propto 1$, we have a
scaling relationship in which all quantiles change to the same extent
as $X$ varies, as in a location/scale family.  While there is no
gaurantee that such a scaling relationship is present, we have found
that it often (approximately) is.  Therefore, we have found it useful
to rotate the CCA solution to identify an approximate scaling
relationship, and then interpret this factor in contrast to the
remaining factors which capture non-scaling effects (e.g.\ effects on
quantiles at different probability points to different extents).

Rotations are an important part of classical factor analysis [ref].
In this spirit, we developed a procedure to rotate the CCA solution so
that the first factor loadings for $Q$ become approximately constant,
reflecting a scaling relationship as discussed above.  This involves
constructing a transformation matrix $F$ and replacing $\eta$ with
$\eta F$, and replacing $\beta$ with $\beta F$.  In doing so, we wish
to preserve the property that the scores $X\beta_j$ remain pairwise
uncorrelated.  It is not possible in general when rotating to also
preserve pairwise uncorrelatedness of the scores $Q\eta_j$, but this
is arguably less important for interpretability.

The rotation algorithm begins by using least squares to regress the
first $Q$-loading vector $\eta_1$ onto a column of $1's$.  Let
$\hat{\eta}_1$ denote the fitted values, and let $f_1$ denote the
coefficient vector such that $\hat{\eta}_1 = \eta f$. The vector $f$
becomes the first column of the transformation matrix $F$.  Subsequent
columns of $F$ are obtained using the Gram-Schmidt procedure to
preserve pairwise uncorrelatedness among the $X$-side scores. [NEEDS
CLEANUP]

As noted above, $Q$ is a matrix of estimated quantiles that are based
partially on $X$.  Our goal is to assess how the underlying true
quantiles $Q_0$ vary with $X$.  To address this question in a way that
overcomes the inbuilt dependence between $Q$ and $X$, we use a
randomization approach.  Specifically, we randomize the rows of $X$
and re-estimate the quantiles in $Q$ and the rotated CCA.  We then
consider the correlations between $Q\eta_j$ and $X\beta_j$ in the
observed and randomized data.  To the extent that the former
correlations are greater, the apparent relationship between $X$ and
$Q$ is unlikely to be solely due to the way in which $Q$ was
constructed.

\subsubsection{NHANES}

We analyzed the NHANES data using the procedures describe above, with
systolic blood pressure as the dependent variable and age, BMI, and
height as explanatory variables.  The explanatory variables are
standardized to have mean zero and unit variance.

Table [nhanes_qnn_drr_table1.tex] contains the results of this
randomization study.  The results show that for both females and for
males, the observed correlations are greater than the randomized
correlations for all factors in the 1 and 2 factor solutions.  In the
3-factor solution, the third factor is no more correlated with the
explanatory variables than under randomization.  Thus we focus on the
2-factor solution below.

The coefficients in table [nhanes_qnn_drr_table1.tex] and the loading
plots in [nhanes_qnn_drr_loadings.pdf] give insight into the
relationship between the explanatory variables and the full
distribution of blood pressure. The rotation procedure successfully
constant an approximately constant loading pattern that becomes the
first factor.  Unsurprisingly, the positive coefficients for age and
BMI indicate that older people and people with higher BMI have
uniformly greater blood pressure quantiles than younger people and
people with lower BMI. Height has a small coefficient for females, but
for males the height coefficient approaches the BMI coeffiicent.

The second factor loading pattern is mostly decreasing and passes
through zero near the median.  A high positive score against this
pattern indicates less dispersion in the outcome around its median,
while a strongly negative score against this pattern indicates more
dispersion in the outcome around its median.  Inspecting the
coefficients we see that older people have more negative scores and
people with higher BMI (and to a lesser extent taller people) have
more positive scores.  This indicates that the conditional quantiles
for older people are more dispered around the conditional median, and
the conditional quantiles for people with higher BMI and who are
taller are less dispersed around the median.  [more could be said
here...]

\subsection{Post-processing QNN estimates with low-rank additive regression}

[...]

\subsection{Post-processing QNN estimates with local regression}
